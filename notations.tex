\documentclass{standalone}

\begin{document}
	\centering
\begin{tabularx}{\textwidth}{|c|X|}
	\hline
	Notations & Descriptions \\
	\hline
	$G \follows \Ger(N, p_A, p_B, p)$ & $G$ est un graphe aléatoire construit selon le modèle \ref{model}  \\
	\hline
	$A:B$ & Désigne une coupe ou une bissection d'un graphe dont $A$ et $B$ forment une partition des sommets \\
	\hline
	$\card{A:B}$ & est le flot entre $A$ et $B$, c'est à dire le nombre d'arêtes entre $A$ et $B$ si le graphe n'est pas pondéré et la somme des capacités de arêtes si le graphe est pondéré. \\
		\hline
	$\preceq$ & désigne la relation d'ordre sur les coupes ou les bissections telle que une coupe est plus petite qu'une autre si son flot est plus petit. \\
		\hline
	$\leq_P$ & désigne la relation d'ordre partielle entre les problèmes par réduction de Karp\cite{21karp}. \\
		\hline
	$\NP, \PP$ & désignent respectivement les classes de complexité polynomiale déterministe et polynomiale non déterministe. \\
		\hline
	$\NPHARD$ & désigne la classe des problèmes au moins aussi durs que tout problème de $\NP$ \\
		\hline
	$\NPCOMPLETE$ & désigne la classe des problèmes de $\NP$ et au moins aussi durs que tous les problèmes de $\NP$, ie $\NPHARD$. \\
		\hline
\end{tabularx}

\end{document}
